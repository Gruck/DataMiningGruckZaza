\section{Traitements préliminaires}

\subsection{Union Européenne}
L'union européenne étant un ensemble de pays qui nous sont assez bien connus, dans leurs grandes caractéristiques économique et sociétales, nous avons estimé qu'il était intéressant de les indiquer de façon claire.\\
Nous avons donc effectué un premier traitement consistant à ajouter un attribut à tous les enregistrements, indiquant si ils appartiennent à l'union européenne ou non.\\

\subsubsection{Valeurs Brutes}
Deux problèmes se posent dans le cas des valeurs brutes : 
\paragraph{Disproportions} Dans le cas de valeurs brutes telles que la surface ou le PIB, des disproportions sont flagrantes et déformes à elle seule l'analyse. Ces critères, quand ils sont utilisés de manière brute deformes les résultats.\\
Pour résoudre ce problème, nous proposons d'établir de nouveaux attributs qui permettent d'obtenir des valeurs relatives : 
\begin{itemize}
	\item Population density 
	\item GDP per inhabitant
\end{itemize}

\paragraph{Normalisation}
Afin de permettre la comparaison des différents enregistrements, et surtout des différents attributs, il est indispensable que ceux-ci soient normalisés.
Dans la suite, nous utiliserons la normalisation par Z-score.







