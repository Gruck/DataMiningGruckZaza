\section{Discrimination des outliers}
Les outliers sont des valeurs qui se situent dans des valeurs extrêmes qui empêchent d'obtenir un modèle cohérent.\\
Il est donc important de les identifier et de les exclure pour la suite de l'étude.

\subsection{Méthode de recherche des outliers}

\paragraph{Box Plot}
Pour rechercher les outliers présents dans le jeu de données, une méthode peut consister à regarder des diagrammes en boites à moustaches.
Les valeurs extrêmes apparaissent clairement.

\begin{figure}[H]
	\begin{center}
		\includegraphics[scale=0.5]{Image/BoxplotOutlierNoMissing2}
		\caption{Diagrammes de boite à moustaches pour le jeu \jeuc}
	\end{center}
\end{figure}

\paragraph{Hierachical trees}
Un autre axe de recherche de ces outliers peut être d'utiliser des clusterisation hiérarchique en mode SINGLE qui font apparaitre les entrées dont les distances sont grandes de façon claires.


\begin{figure}[H]
	\begin{center}
		\includegraphics[scale=0.5]{Image/DendogrammeOutliers}
		\caption{Diagrammes de boite à moustaches pour le jeu \jeuc}
	\end{center}
\end{figure}
%hierachical tree en mode single pour trouver les distances "anormales"

\subsection{Outliers écartés}

\paragraph{Hierachical tree} Ces deux pays doivent être écartés au vu des résultats du clustering hiérarchique. On comprend également que ce sont dans le monde deux exceptions : 
\begin{itemize}
	\item \textbf{USA} : Une taille exceptionnelle, et la richesse de 50 états
	\item \textbf{Singapour} : Cité état à la densité de population et a la richesse par habitant hors norme 
\end{itemize}

\paragraph{BoxPlot}
Le diagramme en boite à moustache montre plusieurs exceptions : 
\begin{itemize}
	\item \textbf{Chad} : Immunisation contre la rubéole exceptionnellement faible
	\item \textbf{Guinée-bisseau} : Il semble difficile de démarrer une entreprise en Guinée-Bissau
	\item \textbf{Luxembourg} : PNB par habitant hors norme
	\item \textbf{Norvège} : PNB par habitant hors norme
\end{itemize}

%liste des outliers écartés