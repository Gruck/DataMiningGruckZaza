\documentclass[a4paper]{article}

\usepackage{hyperref}
%\hypersetup{
%colorlinks=false,              % bool: Liens colorés
%pdfborder={0 0 0}             % Ne pas encadrer les liens
%}
\usepackage[utf8]{inputenc}  
\usepackage[francais]{babel}  
\usepackage[top=2cm, bottom=2cm, left=2cm, right=2cm]{geometry}
\usepackage{graphicx}
\usepackage[final]{pdfpages} 
\usepackage{rotating}
% définir les commandes ici

% s'il y a beaucoup de commandes et de packages à inclure n'h&ésitez pas
% à mettre tout ça dans un fichier include.tex et l'inclure
% \input{include.tex}


\begin{document}
\begin{titlepage}
\begin{center}
 
 \vfill
		\textbf{Hexanome 4211 :} 
		Elisa \bsc{Abidh}, Gaël \bsc{Motte}
		
\vfill
	
		\begin{Huge}
		Data Mining : Rapport \\
		\end{Huge} 

\vfill
	
		\begin{Large}
		
		
		Jeu de données :  \textit{Countries2007}\\
		\end{Large} 


\vfill		
		\begin{Large}
		Mars 2011
		\end{Large}
	\end{center}

\end{titlepage}
%----------------------------------------------------

%--------------------------------- Table des matières
\newpage
\tableofcontents
\newpage
%----------------------------------------------- Plan

\part{Découverte du jeux de données et pré-traitement}

\part{Analyse préliminaires}

\part{Projet - UE et ses nouveaux membres en 2007}

 
%----------------------------------------------------

\end{document}
