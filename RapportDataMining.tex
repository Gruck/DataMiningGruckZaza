\documentclass[a4paper]{article}

\usepackage{hyperref}
%\hypersetup{
%colorlinks=false,              % bool: Liens colorés
%pdfborder={0 0 0}             % Ne pas encadrer les liens
%}
\usepackage[utf8]{inputenc}  
\usepackage[francais]{babel}  
\usepackage[top=2cm, bottom=2cm, left=2cm, right=2cm]{geometry}
\usepackage{graphicx}
\usepackage[final]{pdfpages} 
\usepackage{rotating}
\usepackage{float}

\newcommand{\jeua}{countries2007\_ALL }
\newcommand{\jeub}{countries2007\_NoMissing1 }
\newcommand{\jeuc}{countries2007\_NoMissing2 }
\newcommand{\lesite}{\textit{The world bank}}
% définir les commandes ici

% s'il y a beaucoup de commandes et de packages à inclure n'h&ésitez pas
% à mettre tout ça dans un fichier include.tex et l'inclure
% \input{include.tex}


\begin{document}
\begin{titlepage}
\begin{center}
 
 \vfill
		\begin{Large}\textbf{Hexanome 4211 :} 
		Elisa \bsc{Abidh}, Gaël \bsc{Motte}\end{Large}
		
\vfill
	
		\begin{Huge}
		Data Mining : Rapport \\
		\end{Huge} 

\vfill
	
		 
		\includegraphics[scale=0.3]{Image/drapeaux-du-monde}
\begin{Large}
		
		
		Jeu de données :  \textit{Countries2007}\\
		\end{Large}

\vfill		
		\begin{Large}
		Mars 2011
		\end{Large}
		\includegraphics[scale=0.1]{Image/creative_commons}
	\end{center}
	

\end{titlepage}
%----------------------------------------------------

%--------------------------------- Table des matières
\newpage
\tableofcontents
\newpage
%----------------------------------------------- Plan

\input{intro.tex}

\part{Découverte du jeux de données et pré-traitement}
\section{Découverte}
Le jeu de données traite un bon nombre des indicateurs les plus utilisés lors des analyse géographiques pour de nombreux pays.\\
Ces données proviennent du site \lesite.\\
Sur ce site, nous obtenons d'ailleurs une bonne descriptions de ce que signifient les différents indicateurs. %DONE

%remarques sur le jeu en lui même, la décomposition en
\section{Traitements préliminaires}

\subsection{Union Européenne}
L'union européenne étant un ensemble de pays qui nous sont assez bien connus, dans leurs grandes caractéristiques économique et sociétales, nous avons estimé qu'il était intéressant de les indiquer de façon claire.\\
Nous avons donc effectué un premier traitement consistant à ajouter un attribut à tous les enregistrements, indiquant si ils appartiennent à l'union européenne ou non.\\

\subsubsection{Valeurs Brutes}
Deux problèmes se posent dans le cas des valeurs brutes : 
\paragraph{Disproportions} Dans le cas de valeurs brutes telles que la surface ou le PIB, des disproportions sont flagrantes et déformes à elle seule l'analyse. Ces critères, quand ils sont utilisés de manière brute deformes les résultats.\\
Pour résoudre ce problème, nous proposons d'établir de nouveaux attributs qui permettent d'obtenir des valeurs relatives : 
\begin{itemize}
	\item Population density 
	\item GDP per inhabitant
\end{itemize}

\paragraph{Normalisation}
Afin de permettre la comparaison des différents enregistrements, et surtout des différents attributs, il est indispensable que ceux-ci soient normalisés.
Dans la suite, nous utiliserons la normalisation par Z-score.







 %DONE
%ajout de la classe UE, NON UE.

\section{Adaptations du jeu de données}

Une étape indispensable à la fouille de donnée réside dans le choix des attributs qui seront utilisés pour l'étude.
Il nous font donc y prêter une attention particulière, et choisir ceux qui nous paraissent les plus utiles, et essayer de réduire les dimensions et axes d'études.

\subsection{Corrélation d'attributs}

Afin de réduire les dimensions de l'étude, nous cherchons les corrélations entre les attributs disponibles. Voici les résultats pour le jeu \jeuc.

\begin{figure}[H]
	\begin{center}
		\includegraphics[scale=0.5]{Image/MatriceCorrelationNoMissing2}
		\caption{Matrice de corrélation pour le jeu \jeuc}
	\end{center}
\end{figure}

\subsection{Réduction des dimensions}

Au vu de ces résultat, nous estimons qu'une corrélation de 75\% minimum doublée d'une sémantique forte peut justifier la réduction de dimension. Nous utiliserons pour cela le compostant PCA.
 
Nous avons donc effectué les rapprochements suivants : 
\paragraph{Infrastructure} corrélation : 77\%
		\begin{itemize}
			\item Mobile cellular subscriptions
			\item Internet users
		\end{itemize}
\paragraph{population youth} corrélation : 83\%
		\begin{itemize}
			\item Adolescent fertility rate
			\item Fertility rate
			\item Mortality rate under 5 years
		\end{itemize}
\hfill\\

De plus nous observons que certains attributs sont quelque peu redondants. Par exemple, GDP per Inhabitant (créé par nos soins) et GDI per capita sont corrélé à plus de 99\%. Nous n'utiliserons donc que ce second attribut, donné par \lesite .



 %DONE
%expliquer comment on a reduit la dimmension,
   % methode
   % corélation
   % nouveau nom
   
\section{Discrimination des outliers}
Les outliers sont des valeurs qui se situent dans des valeurs extrêmes qui empêchent d'obtenir un modèle cohérent.\\
Il est donc important de les identifier et de les exclure pour la suite de l'étude.

\subsection{Méthode de recherche des outliers}

\paragraph{Box Plot}
Pour rechercher les outliers présents dans le jeu de données, une méthode peut consister à regarder des diagrammes en boites à moustaches.
Les valeurs extrêmes apparaissent clairement.

\begin{figure}[H]
	\begin{center}
		\includegraphics[scale=0.5]{Image/BoxplotOutlierNoMissing2}
		\caption{Diagrammes de boite à moustaches pour le jeu \jeuc}
	\end{center}
\end{figure}

\paragraph{Hierachical trees}
Un autre axe de recherche de ces outliers peut être d'utiliser des clusterisation hiérarchique en mode SINGLE qui font apparaitre les entrées dont les distances sont grandes de façon claires.


\begin{figure}[H]
	\begin{center}
		\includegraphics[scale=0.5]{Image/DendogrammeOutliers}
		\caption{Diagrammes de boite à moustaches pour le jeu \jeuc}
	\end{center}
\end{figure}
%hierachical tree en mode single pour trouver les distances "anormales"

\subsection{Outliers écartés}

\paragraph{Hierachical tree} Ces deux pays doivent être écartés au vu des résultats du clustering hiérarchique. On comprend également que ce sont dans le monde deux exceptions : 
\begin{itemize}
	\item \textbf{USA} : Une taille exceptionnelle, et la richesse de 50 états
	\item \textbf{Singapour} : Cité état à la densité de population et a la richesse par habitant hors norme 
\end{itemize}

\paragraph{BoxPlot}
Le diagramme en boite à moustache montre plusieurs exceptions : 
\begin{itemize}
	\item \textbf{Chad} : Immunisation contre la rubéole exceptionnellement faible
	\item \textbf{Guinée-bisseau} : Il semble difficile de démarrer une entreprise en Guinée-Bissau
	\item \textbf{Luxembourg} : PNB par habitant hors norme
	\item \textbf{Norvège} : PNB par habitant hors norme
\end{itemize}

%liste des outliers écartés
%DONE
%liste des outliers et raison pour lesquels on les a éliminés


  \newpage 
\part{Analyse du jeu de données}
\section{Axes de Recherche}
expliquer comment on a eu envie d'étudier par rapport aux secteurs 


\section{Nombre de Clusters}
L'analyse hiérarchique va nous permettre de déterminer le nombre de cluster que nous pouvons envisager. Pour ce faire nous utilisons le composant Hierarchical Clustering de Knime avec comme configuration Linkage type : SINGLE.

\paragraph{Indicateur Santé}
Nous réalisons le Hierarchical Clustering avec les attributs santé. Cependant nous décidons d'enlever l'attribut population total car c'est une valeur brute, elle fausse le clustering. Nous obtenons le graphe des distances suivant : 

\begin{figure}[H]
	\begin{center}
		\includegraphics[scale=0.5]{Image/DistanceSanteNoMissing2}
		\caption{Graphe de distance de l'indicateur Santé \jeuc}
	\end{center}
\end{figure}

Suite à l'analyse de ce graphe nous décidons de faire 3 Clusters.

\paragraph{Politique économique}
Nous réalisons la même démarche que précédemment mais avec les attributs de l'indicateur Politique économique. Nous obtenons le graphe de distance suivant : 
\begin{figure}[H]
	\begin{center}
		\includegraphics[scale=0.5]{Image/DistancePolitiqueNoMissing2}
		\caption{Graphe de distance de l'indicateur Politique économique\jeuc}
	\end{center}
\end{figure}

Après étude du graphe nous ferons 4 clusters.
%montrer les résultats  
   % dendrogramme
   % distance
   % conclusion sur le nb de clusters
\section{Clustering par partionement}
Nous avons réalisé le clustering par partitionnement en mettant en place deux méthodes : hierarchique et non hierarchique. 

\subsection{Santé}

\paragraph{Hierarchique}
Nous avons utilisé le composant hierarchical Clustering avec la configuration Linkage type : COMPLETE. Pour observer les résultats nous avons utilisé le composant Scatter Matrix.
Nous avions choisi de faire 3 clusters (cf section précédente), dont la répartitions des pays se fait comme suit : 

\begin{figure}[H]
	\begin{center}
		\includegraphics[scale=0.5]{Image/TableViewSanteNoMissing2}
		\caption{Liste des pays par clusters sur les critères de santé avec le jeu \jeuc}
	\end{center}
\end{figure}


Nous obtenons les résultats suivants : 

\begin{figure}[H]
	\begin{center}
		\includegraphics[scale=0.5]{Image/ScatterMatrixSanteNoMissing2}
		\caption{Scatter Matrix des attributs de l'indicateur Santé \jeuc}
	\end{center}
\end{figure}


\paragraph{Non Hierarchique}
L'application de K-means avec 99 itérations donne les clusters suivants: 

\begin{figure}[H]
	\begin{center}
		\includegraphics[scale=0.5]{Image/TableViewSanteKmeansNoMissing2}
		\caption{Liste des pays par clusters sur les critères de santé avec le jeu \jeuc}
	\end{center}
\end{figure}


Nous obtenons les résultats suivants : 

\begin{figure}[H]
	\begin{center}
		\includegraphics[scale=0.5]{Image/scattermatrixSantekmeansNoMissing2}
		\caption{Scatter Matrix des attributs de l'indicateur Santé \jeuc}
	\end{center}
\end{figure}

K-means n'étant pas déterministe, nous devons vérifier sa stabilité.\\
Nous effectuons donc plusieurs essais de partitionement avec cet algorithme. Nous voyons de façon très claire que les clusters ne sont pas stables  
\begin{figure}[H]
	\begin{center}
		\includegraphics[scale=0.4]{Image/ScatterPlotSanteKmeansStabiliteNoMissing2}
		\caption{Evolution des clusters au cours des itérations}
	\end{center}
\end{figure}


\paragraph{Analyse}
Nous pouvons constater avec l'utilisation des deux méthodes de classification que nous avons une différence mais qu'elle n'est pas flagrante. Nous retrouvons les mêmes tendances.\\
Nous remarquons que nos clusters sont plus ou moins justifiable. En effet dans certain cas nous repérons bien 3 clusters distinct comme avec les attributs : Fertility rate and mortality under 5 et Life expectancy at birth. Mais cependant dans bon nombre de cas nous repérons plutôt un nuage de point très dispersés ou qui pourrait être modélisé comme une droite.\\
A la vue des représentations on constate que on aurait peut être du ne faire que deux clusters. Pour le moment nous ne pouvons pas conclure. Nous pouvons constater que le cluster vert est bien distinct du cluster bleu et que ces indicateurs de santé montrent de meilleures conditions  sanitaires.\\
Comme le montre le tableau plus haut nous remarquons que le cluster vert regroupe bon nombre de pays européen alors que le cluster bleu regroupe bon nombre de pays d'Afrique ce qui est un résultat pas cohérent en vue de ce que nous savons de l'état actuel de ces pays.


\subsection{Économie}
Au vu des distances obtenues dans la section précédente, nous obtenons 4 clusters.

\paragraph{Hierarchique} Par classification hiérarchique, nous obtenons les clusters suivants :

\begin{figure}[H]
	\begin{center}
		\includegraphics[scale=0.5]{Image/TableViewPolitiqueNoMissing2}
		\caption{Liste des pays par clusters sur les critères économiques avec le jeu \jeuc}
	\end{center}
\end{figure}


Nous obtenons les résultats suivants : 

\begin{figure}[H]
	\begin{center}
		\includegraphics[scale=0.5]{Image/ScatterMatrixPolitiqueNoMissing2}
		\caption{Scatter Matrix des attributs de l'indicateur de politique économique \jeuc}
	\end{center}
\end{figure}

\paragraph{Non Hierarchique}
L'application de K-means donne les clusters suivants: 

\begin{figure}[H]
	\begin{center}
		\includegraphics[scale=0.5]{Image/TableViewPolitiqueKmeansNomissing2}
		\caption{Liste des pays par clusters sur les critères de santé avec le jeu \jeuc}
	\end{center}
\end{figure}


Nous obtenons les résultats suivants : 

\begin{figure}[H]
	\begin{center}
		\includegraphics[scale=0.5]{Image/scattermatrixPolitiquekmeansNoMissing2}
		\caption{Scatter Matrix des attributs de l'indicateur Santé \jeuc}
	\end{center}
\end{figure}

Nous devrions étudier la aussi la stabilité du clustering obtenu par K-means, mais comme dans les section précédente, les clusters obtenus ne sont pas stables.

\paragraph{Analyse}
De même que précédemment nous remarquons que quelque soit la méthode de classification que nous utilisons nous obtenons les mêmes tendances.\\
Pour cet indicateur nous obtenons des graphes pas très parlant. Nous remarquons cependant que pour certains graphes le nombre de valeurs sont très larges (GNI per capita et GDP growth par exemple) ce qui met en avant de grande disparité entre les pays. Il aurait peut être était intéressant de se concentrer sur moins de pays pour cet indicateur.\\ test





% liste des attributs qui servent a partitioner
   % pourquoi
   % resultats

\newpage
\part{Projet - UE et ses nouveaux membres en 2007}
\section{Introduction}
En 2007, année de collecte des valeurs du jeu \jeuc, l'union européene s'est élargie avec l'entrée de de la Bulgarie et de la Roumanie. Cette décisions fait probablement suite à une étude. Celle-ci a probablement conclu que ces deux pays étaient suffisement "proches" des autres pays de l'UE pour ne pas causer trop de problèmes.\\
Nous tenterons donc de vérifier sur certains critères si ces pays sont significativement différents, ou similaires, du reste de l'UE.

 
% idéee de justification de l'idée
\section{Méthodologie suivie}

\subsection{Première analyse sur le jeu \jeuc}

\subsection{Analyse sur le jeu \jeua}
Pour la suite du projet il aurait été intéressant de tester l'apprentissage. Nous nous serions limité uniquement au pays qui aurait les attributs qui nous intéresse renseignés. Nous n'aurions pas pris les pays avec des attributs manquant car comme nous ne savons pas exactement ce que fais knime, nous ne serions pas les exploiter correctement. Cependant en vue de nos résultats qui ne sont pas très concluant et du manque de temps nous n'avons pas eu le temps de tester sur le jeu \jeua . Si nous avions le temps de le faire nous aurions d'abord commencer par le \jeub . Pour se faire nous aurions fais l'analyse sur les attributs santé puis sur ceux de l'indicateur économique politique. Cette étape nous aurait permis de voir si le fait de supprimer les pays non renseignés pour ce qui nous intéresse est un bon choix et deuxièmement de vérifier si notre méthode est faisable sur un jeu de donnée plus grand.



% reprise de ce qui été produit

\section{Validation des modèles}


\subsection{Classification supervisée}
Afin de mettre place des modèles utilisables et de qualité certaine, nous utiliserons la classification supervisée.\\
Nous nous attaquerons donc en priorité à l'union européenne et tacherons de vérifier si les résultats sont probants.

\paragraph{Global}
Dans un premier temps, nous étudierons si il est possible de reconnaitre un pays de l'Union Européenne en se basant sur l'ensemble des critères disponibles dans le jeu \jeuc .

\paragraph{Santé}
Il nous semble intéressant de vérifier si l'on peut déterminer si un pays est membre de l'UE en se sur les seuls critères de santé.\\
Nous nous appuierons donc sur des arbres de décisions pour permettre cet apprentissage.
\begin{figure}[H]
	\begin{center}
		\includegraphics[scale=0.5]{Image/DecisionTreeSanteBlugarieRoumanie}
		\caption{Resultats de la CrossValidation de l'apprentissage des Membres de UE sur les critères de santé}
	\end{center}
\end{figure}



   % cross validation
   
\section{Résultats obtenus}
En appliquant le même arbre de décision, nous obtenons les résultats suivants
\begin{figure}[H]
	\begin{center}
		\includegraphics[scale=0.5]{Image/ErrorRatesSante}
		\caption{Resultats de la prédiction appliquée à la Roumanie et à la Bulgarie}
	\end{center}
\end{figure}
% présentation des resultats
\section{Conclusion sur le projet}
TODO

% analyse critique de ce qu'on a fait 

\part{Annexes : Workflows}


 
%----------------------------------------------------

\end{document}
