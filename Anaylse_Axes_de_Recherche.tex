\section{Axes de Recherche}
Après réflexion nous avons décidé de regrouper les attributs par catégorie d'indicateur. En effet nous préférons faire du clustering sur un groupe d'attribut plutôt que sur la totalité des attributs.
Pour faire ces groupes d'attributs nous avons pris comme référence le site de World Bank qui a lui même récolté les données. Nous avons dégagé deux indicateurs principaux : Santé et Politique économique. Pour le jeu de donnée 'no missing2', l'indicateur Santé regroupe les attributs suivant : Adolescent fertility rate, Fertility rate total, Immunization measles, Life expectancy at birth, Mortality rate under-5, Population growth et Population total; l'indicateur Politique économique regroupe lui les attributs: GDP, GDP growth, GNI per capita, atlas method et GNI, atlas method.\\
Les autres attributs ne seront pas traité dans la suite de l'analyse, nous avons fais le choix pour la suite de nous centrer uniquement sur les attributs cités plus haut.\\
Nous allons par la suite faire du clustering avec seulement les attributs de l'indicateur Santé puis uniquement les attributs Politique économique.
Le but cette réduction d'attributs est de voir si l'indicateur Santé et Politique économique permet de dégager des groupes de pays, de voir ou se situe l'Europe. Par la suite nous souhaitons étendre la recherche sur le grand jeu de donné. Pour le grand jeu de donnée nous supprimerons tous les pays dont les attributs qui nous intéresse ne sont pas renseignés. Nous justifions ce choix par le fait que nous ne savons pas ce que fais exactement Knime avec les données non renseignées. 