\section{Méthodologie suivie}

\subsection{Première analyse sur le jeu \jeuc}
Le jeu \jeuc comence à nous être familier. Bien qu'aucun partitionement n'ai été trouvé, nous tenterons de vérifier si il est possible \textit{d'apprendre} ce qu'est l'Union Européenne sur différents critèreres.\\ Nous tacherons alors de vérifier si les deux nouveaux membres en 2007, que sont la Bulgarie et la Roumanie, correspondent bien à cette image des pays membres.

\subsection{Analyse sur le jeu \jeua}
Pour la suite du projet il aurait été intéressant de tester l'apprentissage. Nous nous serions limité uniquement au pays qui aurait les attributs qui nous intéresse renseignés. Nous n'aurions pas pris les pays avec des attributs manquant car comme nous ne savons pas exactement ce que fais knime, nous ne serions pas les exploiter correctement. Cependant en vue de nos résultats qui ne sont pas très concluant et du manque de temps nous n'avons pas eu le temps de tester sur le jeu \jeua . Si nous avions le temps de le faire nous aurions d'abord commencer par le \jeub . Pour se faire nous aurions fais l'analyse sur les attributs santé puis sur ceux de l'indicateur économique politique. Cette étape nous aurait permis de voir si le fait de supprimer les pays non renseignés pour ce qui nous intéresse est un bon choix et deuxièmement de vérifier si notre méthode est faisable sur un jeu de donnée plus grand.


